\documentclass[a4paper, 11pt]{article}

\usepackage{amssymb}
\usepackage{proof}
\usepackage{color}

\title{E-Prover's output format specification}
\author{Giselle Reis, Stephan Schultz and whoever
else wants to help ;-)}

\begin{document}
\maketitle

E-prover reasons on equality atoms. It transforms every atom into an equality
expression, i.e., $P \equiv P = \top$ and $\neg P \equiv P \neq \top$.
%
In this document we try to give a quick explanation of the inference rules used
and theirs names in the output proofs. In all rules below, $C, D$ denote a
clause of the form $L_1 \vee ... \vee L_n$, where each $L_i$ (or $L$) is an
equality literal. $F$ denotes an arbitrary formula.

\begin{itemize}
\item \texttt{er}: 
  Equality resolution: $x \neq a \vee x = x \Rightarrow a = a$

  \[
  \infer[\mathtt{er}]{\sigma(C)}{u \neq v \vee C}
  \]

  Where $\sigma = mgu(u,v)$.

\item \texttt{pm}: 
  Paramodulation. Note that E considers all literals as
  equational, and thus also performs resolution by a combination of top-level
  paramodulation and (implicit) clause normalization.

  \[
  \infer[\mathtt{pm?} + \mbox{ implicit factoring}]{C \vee L[s] \vee D}{
    C \vee L[t] &
    t = s \vee D
  }
  \]

\item \texttt{spm}: 
  Simultaneous Paramodulation. This is similar to
  paramodulation, but replaces all instances of the same term in the original
  clause in one step.

  \[
  \infer[\mathtt{spm?} + \mbox{ implicit factoring}]{C[s] \vee D}{
    C[t] &
    t = s \vee D
  }
  \]

\item \texttt{ef}: 
  Equality factoring (factor equations on one side only, move
  the remaining disequation into the precondition): 
  \newline
  $x = a \vee b = c \vee x = d \Rightarrow a \neq c \vee b = c \vee b = d$

  \[
  \infer[\mathtt{ef}]{\sigma(t \neq v \vee u = v \vee C)}{
    s = t \vee u = v \vee C
  }
  \]
  Where $\sigma = mgu(s,u)$.

\item \texttt{apply\_def}: 
  Apply a definition, usually replacing a complex
  formula by a single literal.

  \[
  \infer[\mathtt{apply\_def}]{C \vee F}{
    C \vee def(F)
  }
  \]

\item \texttt{introduced(definition)}: 
  Introduce a new definition.

  \[
  \infer[\mathtt{apply\_def}]{C \vee def(F)}{
    C \vee F
  }
  \]

\item \texttt{rw}: 
  Rewriting, each rw-expression corresponds to exactly one
  rewrite step with the named clause. This is also used for equational unfolding.

  \[
  \infer[\mathtt{rw}???]{L[s] \vee C}{
    t = s &
    L[t] \vee C
  }
  \]
  
  {\color{red}Difference from paramodulation?}

\item \texttt{sr}: 
  Simplify-reflect: An (equational) version of unit-cutting. As
  as example, see this positive simplify-reflect step: 
  \newline
  $[a = b], [f(a) \neq f(b)] \Rightarrow \Box $

  \[
  \infer[\mathtt{sr}]{C \vee D}{
    C \vee a = b &
    L[a] \neq L[b] \vee D
  }
  \]
  
  Given that $a$ and $b$ occur in the same positions in $L[a]$ and $L[b]$.

\item \texttt{csr}: 
  Contextual simplify-reflect (also known as contextual
  literal cutting or subsumption resolution): Literal cutting with non-unit
  clauses if one of the literals is implied in the context of the clause to be
  simplified.

  \[
  \infer[\mathtt{csr}]{C \vee D}{
    C \vee a = b &
    L[a] \neq L[b] \vee D
  }
  \]

  {\color{red}Rule deduced from the examples, but it turns out to be the same as
  \texttt{sr}. I am missing something.}

\item \texttt{ar}: 
  AC-resolution: Delete literals that are trivial modulo the
  AC-theory induced by the named clauses.

  \[
  \infer[\mathtt{ar}]{C}{C \vee D}
  \]

  Given that $C, \mbox{ associativity, commutativity } \vdash D$.

\item \texttt{cn}: 
  Clause normalize, delete trivial and repeated literals. This
  step is often implicit, but can sometimes occur explicitly.

  \[
  \infer[\mathtt{ar}]{C}{C \vee D}
  \]

  Given that for every $L \in D. L \in C \vee L \equiv x \neq x$.

\item \texttt{condense}:
  Apply condensation (replace the clause by one of its factors that is more
  general).

  \[
  \infer[\mathtt{condense}]{C}{C \vee L_1 \vee ... \vee L_n}
  \]

  Given that for every $L_i$, there exists an $L \in C$ such that $L\sigma =
  L_i$ for some substitution $\sigma$.

\end{itemize}

Additionally, the clausification will use additional rule names:

\begin{itemize}
\item \texttt{assume\_negation}:
  Negate a conjecture for a proof by refutation. 

\item \texttt{fof\_nnf}:
  Convert a formula to negation normal form by moving negation signs inward to
  the literals, and eliminating equivalences and implications.

\item \texttt{shift\_quantors}:
  Move quantors (inwards for mini-scoping or outwards for the final conjunctive
  normal form).

\item \texttt{variable\_rename}:
  Rename bound variables to make sure each quantor binds a different variable.

\item \texttt{skolemize}:
  Eliminate existential quantors via Skolemization. 

\item \texttt{distribute}:
  Apply distributivity law to move and outwards over or.

\item \texttt{split\_conjunct}:
  Split of one conjunct from a formula in conjunctive normal form to create a
  clause.

\item \texttt{fof\_simplification}:
  Perform standard simplification steps on the formula. This may also replaces
  back implications and exclusive or with equivalent constructs using
  implications and equivalences.

\end{itemize}


\end{document}
